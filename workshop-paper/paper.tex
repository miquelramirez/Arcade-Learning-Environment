  
\documentclass[letterpaper]{article}
\usepackage{aaai}
\usepackage{times}
\usepackage{helvet}
\usepackage{courier}
\frenchspacing
\setlength{\pdfpagewidth}{8.5in}
\setlength{\pdfpageheight}{11in}


\newcommand{\tuple}[1]{{\langle #1\rangle}}
\newcommand{\triple}[1]{{\langle #1\rangle}}
\newcommand{\pair}[1]{{\langle #1\rangle}}

\newcommand{\Omit}[1]{}

\newcommand{\eqdef}{\stackrel{\hbox{\tiny{def}}}{=}}
\newcommand{\IW}{{\textit{IW}}}
\newcommand{\SIW}{{\textit{SIW}}}
\newcommand{\ID}{{\textit{ID}}}
\newcommand{\BRFS}{{\textit{BrFS}}}

% AAAI (comment for ECAI)
% \newtheorem{theorem}{Theorem}


 \newtheorem{theorem}{Definition}
% \newtheorem{proposition}[theorem]{Proposition}
% \newtheorem{lemma}[theorem]{Lemma}
% \newtheorem{corollary}[theorem]{Corollary}
 \newtheorem{definition}[theorem]{Definition}


 \begin{document}
% The file aaai.sty is the style file for AAAI Press 
% proceedings, working notes, and technical reports.
%
\pdfinfo{
/Title (Classical Planning Algorithms on the Atari Video Games: Preliminary Results)
/Author (Nir Lipovetzky, Miquel Ramirez, Hector Geffner)}

\setcounter{secnumdepth}{0}  

\title{Classical Planning Algorithms � on the Atari Video Games: Preliminary Results}

\author{Nir, Miquel, H}
%AAAI Press\\
%Association for the Advancement of Artificial Intelligence\\
%2275 East Bayshore Road, Suite 160\\
% Palo Alto, California 94303\\}

\maketitle


%% Pending: randomization, discount factor, iw11 ..
%% iw2 multivalued expands up to 256^2 n^nodes, while iw2 boolean 16^2 n^2 nodes; can iw2 boolean beat iw1 multivalued?
%% Important. optimization, need Dikjstra por postprocessing .. see emails 


\begin{abstract}
\begin{quote}
The Atari  2600 games supported in  the Arcade Learning Environment (Bellemare et. al. 2013) 
all feature a known initial (RAM) state and actions that have deterministic effects. Classical planners, however, 
cannot be used  for selecting actions  for  two reasons: first,  no  compact PDDL-model of  the games is given, 
and  more importantly, the actions effects and the goals are  not known a priori. Moreover, in these games there is usually  
no set of goals to be achieved but rewards to be collected. These features do not preclude the use of   classical  algorithms like
breadth-first search or Dijkstra's algorithm,  but    these  algorithms are not effective over  large state spaces. We thus turn to a
different class of classical planning algorithms  introduced recently  that perform a \emph{structured exploration} of the state space; namely,  
like breadth-first search and Dijkstra's  algorithm they are ``blind'' and hence  do not require prior knowledge of state transitions, 
costs (rewards) or goals, and yet, like  heuristic search algorithms, they have been shown to be   effective for solving   problems over huge 
state spaces. The simplest such algorithm, called Iterated Width or IW, consists of  a sequence of calls  IW(1), IW(2), .., IW(k) where  IW(i) is  
a breadth-first search in which a state is  pruned when it is not the first state in the search  to make true some set of i atoms or less.
Preliminary empirical results suggest that the performance of IW with the  $k$ parameter fixed to $1$, i.e., IW(1), 
is much better than random or breadth-first search, and close to the performance of UCT. Other simple variations of IW
that combine exploration and exploitation are evaluated as well.
\end{quote}
\end{abstract}

\section{Introduction}


The Arcade Learning Environment (ALE)  provides a challenging  platform for evaluating general, domain-independent AI planners and 
learners through a convenient interface to hundreds of Atari 2600 games \cite{bellemare:jair2013}. 
Results have been reported so far for basic planning algorithms like breadth-first 
search and UCT \cite{bellemare:jair2013}, and for  reinforcement learning
and evolutionary  methods \cite{bellemare:jair2013,deep-mind-atari,risto-texas:atari-games}.
The empirical results are impressive in some cases, yet a lot remains to be done, as no method  approaches   the 
performance of  human players across a broad range of games.

While all  these games feature a known initial (RAM) state and actions that
have deterministic effects on this state, the problem of selecting the next action
to be done cannot be addressed with state-of-the-art classical planners for two 
reasons. First, there is no compact PDDL-like encoding of the domain, and more 
importantly,  the goal to be achieved in each game  is not given. Indeed, there are 
often no goals at all but rewards $r(a,s)$ that potentially depend on the action $a$ and 
the state $s$ in  which  the action is performed, and these rewards are not known a priori either. 
Thus no model of the  goals  or the  rewards can be used to bias the search, and hence, 
standard heuristic search and classical planning methods can't be used. 

The action selection problem in the Atari games can be seen as a \emph{reinforcement learning} problem 
\cite{sutton:book} over a deterministic MDP where the state transitions and rewards are not known, or alternatively,
as a  \emph{net-benefit planning problem} \cite{ref-net-benefit-planning} with unknown state transitions  and  rewards. Namely, 
we seek  an action sequence $a_0, a_1, \ldots, a_m$ from the given current  state $s_0$ that generates 
a state sequence $s_0, s_1, \ldots, s_{m+1}$ with  maximum  total reward $\sum_{i=0}^{m} r(a_i,s_i)$,  
where $m$ is a sufficiently large planning horizon, and the rewards $r(a_i,s_i)$ and the  state-transitions $s_{i+1}=f(a_i,s_i)$ 
are known only after the action $a_i$ is applied in the state $s_i$. 

The presence of unknown transition and  rewards in  the Atari games does not preclude the use of blind-search
methods like breadth-first search, Dijkstra's algorithm \cite{dijkstra},  or learning methods  such as 
LRTA* \cite{korf:lrta}, UCT \cite{uct}, and Q-learning  \cite{sutton:rl,bertsekas:neuro}. Indeed, 
the net-benefit planning problem with unknown state transitions and rewards over a given planning
horizon, can be mapped into a standard \emph{shortest-path problem} which can be solved optimally by Dijkstra's algorithm.
For this, we just need to  map the unknown rewards $r(a,s)$ into positive (unknown) action costs $c(a,s) = C - r(a,s)$
where  $C$ is a large constant that exceeds the maximum possible reward. The fact that the state 
transition and cost functions  $f(a,s)$ and $c(a,s)$ are not known a priori doesn't affect the applicability of 
Dijkstra's algorithm, which requires the value of these functions precisely when  the action $a$
is applied in the state $s$. 

The limitation of the   basic blind search  methods is that they are not effective  over very spaces, neither
for solving problems off-line, nor for guiding  a lookahead search for solving problems on-line. 
In this work, we thus turn to a recent class of  planning algorithms that combine the scope of blind search
methods with the performance of state-of-the-art classical planning algorithms: namely, like ``blind'' search algorithms
they do not require prior knowledge of state transitions, costs, or goals, and yet like heuristic algorithms
they manage to search large  state spaces effectively. The basic algorithm in this class is 
called IW for Iterated Width search \cite{nir:ecai2012}. IW consists of a sequence of calls  IW(1), IW(2), .., IW(k), where 
IW(i) is  a  standard breadth-first search where states are pruned right away when they fail to make true some new tuple (set) 
of at most $i$ atoms. Namely,   IW(1) is a breadth-first search that keeps  a state only if  the state is   the first  one 
in the search to make some atom  true; IW(2) keeps a state  only if the state is  the first one to  make  a pair of atoms true, and 
so on.  Like plain breadth-first  and iterative deepening searches,  IW  is 
complete, while searching the state space in a way that makes use of the 
\emph{structure of states} given by the values of a finite set of state variables.
In the Atari games, the (RAM) state is given by a vector of 128 bytes, which we associate with 128 variables 
$X_i$, $i=1, \ldots, 128$, each of which may take up to $256$ values $x_j$. A state $s$ makes an atom $X_i=x_{j}$ 
true when the value of the $i$-th byte in the state vector $s$ is $x_j$. 

While a normal, complete breadth-first search runs in time that is exponential in the total number of variables $X_i$, 
the procedure $IW(k)$ runs in time that is exponential in the $k$ parameter, $1 \leq k \leq n$. 
Elsewhere, it has been shown that  IW exhibits   state-of-the-art performance on most existing planning benchmark domains 
when goals are restricted to be single atoms  where IW provably runs in low polynomial time in the number of
variables \cite{nir:ecai2012}. For example, in a a block world instance with any number of blocks and any initial
configuration,  the goal of  placing a given  block on top of another, will be achievable in  quadratic time  
in the number of blocks as any such instance can be shown to  have \emph{width} 2. This means  that IW(2) will be 
not only complete then, but also optimal. This is also true for most other benchmark domains in classical planning
that turn out to be have small, bounded widths for any instance, as long as the goal is atomic. 
For dealing with the benchmarks where goals are  not single atoms, e.g., where a given \emph{tower} of blocks needs to be built, 
simple extensions of the basic IW procedure have been developed as well, such as Serialized IW, where IW is 
used to achieve the goals one at at time \cite{nir:ecai2012,nir:ecai2014}. The extension of IW for problems where there is no goal but an
additive reward measure   to be optimized is direct.

%%  Each iteration IW(k) keeps track of the accumulated reward up to each state $s$ that has been
%% generated but  not pruned, as $R(s) = R(s')+ r(a,s')$ where $s'$ and $a$ are  the single parent state
%% and action $a$ that led to $s$ in IW(k) with $R(s)=0$ for the current state $s$. The best reward 

The sequence of calls IW(1), IW(2), \ldots, IW(k) manages to combine the scope of ``blind'' search algorithms with the performance of 
heuristic-search algorithms by exploring states  in an  order dictated by the ``width''  parameter $k$; 
states that have width 1 are explored first in linear time, then states that have  width two are explored in quadratic time, and so on. 
In classical planning problems, if the goal has a small width (something common in the classical planning domains when the  goal is a single atom, as
we have seen above), IW will find goal states  in low polynomial time. Similarly, in the reward-based setting, if  there are states of high reward 
that have small width, IW will find such states in low polynomial time, even if the state space is exponential in the number of variables. 
In this paper, we aim to test whether these ideas work for the Atari games too  when  the RAM state of the emulator
is used  as the system state. 

The paper is organized as follows. We review the Iterated Width  algorithm and where it comes from, 
look at  some simple   variations of the basic algorithm that appear to be convenient  for the Atari
games, present and analyze the preliminary experimental results, summarize the main lessons, and
discuss related and future work.


\section{Iterated Width}

The Iterative Width  (IW) algorithm has been introduced as a classical planning algorithm
that takes a planning problem as an input, and computes an action sequence that solves the problem 
as the output \cite{nir:ecai2012}. The algorithm however
applies to a broader range of problems. We will characterize such
problems  by means  a finite and discrete set of states (the state space)
that correspond to vectors of size $n$. Namely, the 
states are  \emph{structured} or \emph{factored} , and we take each of the locations in the vector
to represent a variable $X_i$, and the value at that vector location to 
represent the value $x_j$ of variable $X_i$ in the state. 
In addition to the states space, a problem is defined by an initial state $s_0$, 
a set of actions $A(s)$ applicable in each state, a 
transition function $f$ such that  $s' = f(a,s)$ is 
the state that results from applying action $a \in A(s)$ to the state $s$, 
and rewards $r(a,s)$ represented by real numbers that result from  applying action $a$ in state $s$.
The transition and reward functions do not  need to be known a priori, yet in that case, the  
state and  reward that results from the application of an action in a state needs to be \emph{observable}.
The task is to compute an action sequence $a_0, \ldots, a_m$ for a large horizon $m$
that generates a state sequence $s_0, \ldots, s_{m+1}$ that  maximizes the accumulated 
reward $\sum_{i=0}^{m} r(a_i,s_i)$, or that provides a good approximation.

\subsection{The Algorithm}

IW consists of a sequence of calls  $\IW(i)$ for $i=0, 1, 2, \ldots$
over a problem $P$ until a termination condition is reached. 
The procedure $\IW(i)$  is a plain forward-state \emph{breadth-first search} with just one change: right after a state 
$s$ is generated, the state is  pruned if it doesn't pass a simple \emph{novelty test}.
More precisely,

\begin{itemize}
\item The \emph{novelty of a newly generate state $s$  in a  breadth-first search}
is   $1$ when $s$ is the first state generated in the search that makes true an atom $X=x$,
$2$  when $s$ is the first state that makes a pair of atoms $X=x$ and $Y=y$ true, and so on. 

\item  $\IW(i)$ is a breadth-first search that prunes newly generated states when their novelty measure
is greater than $i$.

\item $\IW$  calls $\IW(i)$ sequentially for $i=1,2,\ldots$ until a termination condition is reached, 
returning then the best path found.
\end{itemize}

For classical planning, the termination condition is the achievement of the goal. In the on-line setting, 
as in the Atari games, the termination condition is given by a time window or a maximum number of generated nodes.
The  best path found by IW is the path  that has a  maximum accumulated reward. 
The accummulated reward $R(s)$  of a state $s$ reached in an iteration of IW is determined by the  unique parent state $s'$ 
and action $a$ leading to $s$ from $s'$ as $R(s) = R(s') + r(a,s')$. The best state is the state $s$ with maximum reward
$R(s)$ generated but not pruned by IW, and the best path is the one that leads to the state $s$ from the current state.
The action selected to be done next is then the first action along such a path  (** see eventually randomization **).

%% Like in standard breadth-first search, each iteration IW(i)  prunes 
%%% the  duplicate states as well.

\begin{table}
\begin{center}
%\resizebox{3in}{!}{
{\scriptsize
% \begin{tabular}{r@{}lrrrr@{}}
\begin{tabular}{llcccc}
%% table header
\hline\\[-1.3ex]
%\toprule
\# & \multicolumn{1}{c}{Domain} & \multicolumn{1}{c}{I} & \multicolumn{1}{c}{IW(1)} & \multicolumn{1}{c}{IW(2)} & \multicolumn{1}{c}{Neither} \\[.2ex]

\hline\\[-1.3ex]
%\hline
%% table body
1. & 8puzzle	&400	&55\%	&45\%	&0\%\\
2. & Barman	&232	&9\%	&0\%	&91\%\\
3. & Blocks World	&598	&26\%	&74\%	&0\%\\
4. &Cybersecure	&86	&65\%	&0\%	&35\%\\
% 5. &Depots	&189	&11\%	&66\%	&23\%\\
% 6. & Driver	&259	&45\%	&55\%	&0\%\\
% 7. & Elevators	&510	&0\%	&100\%	&0\%\\
% 8. & Ferry	&650	&36\%	&64\%	&0\%\\
% 9. & Floortile	&538	&96\%	&4\%	&0\%\\
% 10. & Freecell	&76	&8\%	&92\%	&0\%\\
% 11. & Grid	&19	&5\%	&84\%	&11\%\\
% 12. & Gripper	&1275	&0\%	&100\%	&0\%\\
% 13. & Logistics	&249	&18\%	&82\%	&0\%\\
% 14. & Miconic	&650	&0\%	&100\%	&0\%\\
% 15. & Mprime	&43	&5\%	&95\%	&0\%\\
% 16. & Mystery	&30	&7\%	&93\%	&0\%\\
% 17. & NoMystery	&210	&0\%	&100\%	&0\%\\
% 18. & OpenStacks	&630	&0\%	&0\%	&100\%\\
% 19. & OpenStacksIPC6	&1230	&5\%	&16\%	&79\%\\
\ldots & \ldots & \ldots & \ldots & \ldots & \\ 
20. & ParcPrinter	&975	&85\%	&15\%	&0\%\\
21. & Parking	&540	&77\%	&23\%	&0\%\\
22. & Pegsol	&964	&92\%	&8\%	&0\%\\
23. & Pipes-NonTan	&259	&44\%	&56\%	&0\%\\
24. & Pipes-Tan	&369	&59\%	&37\%	&3\%\\
25. & PSRsmall	&316	&92\%	&0\%	&8\%\\
26. & Rovers	&488	&47\%	&53\%	&0\%\\
27. & Satellite	&308	&11\%	&89\%	&0\%\\
28. & Scanalyzer	&624	&100\%	&0\%	&0\%\\
29. & Sokoban	&153	&37\%	&36\%	&27\%\\
30. & Storage	&240	&100\%	&0\%	&0\%\\
31. & Tidybot	&84	&12\%	&39\%	&49\%\\
32. & Tpp	&315	&0\%	&92\%	&8\%\\
33. & Transport	&330	&0\%	&100\%	&0\%\\
34. & Trucks	&345	&0\%	&100\%	&0\%\\
35. & Visitall	&21859	&100\%	&0\%	&0\%\\
36. & Woodworking	&1659	&100\%	&0\%	&0\%\\
37. & Zeno	&219	&21\%	&79\%	&0\%\\[.2ex]
%\hline
\hline\\[-1.3ex]
& Total/Avgs&37921	&37.0\%	&51.3\%	&11.7\%\\
\hline
\end{tabular}
}

\vskip .1cm


{\scriptsize
\begin{tabular}{@{}ccccc@{}}
\hline\\[-1.3ex]
\multicolumn{1}{c}{\# Instances}  &\multicolumn{1}{c}{\IW} & \multicolumn{1}{c}{\ID} & \multicolumn{1}{c}{\BRFS} &\multicolumn{1}{c}{\textit{GBFS +} $h_{add}$} \\[.2ex]
\hline\\[-1.3ex]
37921  &34627 &9010	&8762 &34849 \\
\hline
%\hline
\end{tabular}
}

\end{center}

\caption{\small 
\emph{Top:} Number of classical planning problems  and percentages solved by IW(1), IW(2), or neither. Instances obtained
by splitting benchmarks  with $N$ atomic goals into $N$ problems with atomic goals.
\emph{Bottom:} Number of instances solved by \IW\ in comparison with  Iterative Deepening (\ID),  
Breadth-First Search (\BRFS),   and Greedy Best First Search (\textit{GBFS}) guided by  additive heuristic.
Time and  memory outs  after 30 minutes and 2 GB resp. From \cite{nir:ecai2102}.
}
\label{table1}
\end{table}
%%%%
%%%

\subsection{Performance and Width}

IW is a systematic and complete blind-search algorithm like breadth-first search (BrFS) and iterative deepening (ID),
but   unlike these algorithms, it uses the factored representation of the states in terms of variables to structure the 
search in a different way. This structured exploration  has proved to be very effective in classical planning 
over most of the benchmark domains  when the goals are restricted to be single atoms. Table~\ref{table1} from \cite{nir:ecai2012} shows 
the percentage of instances that are  solved by  the first iteration of IW, i.e.  IW(1), by the second iteration 
IW(2), and by neither one.  These are instances that  have been obtained from the existing benchmarks
by splitting problems with $N$ atomic goals, into $N$ problems with one atomic goal each. As the table shows, 
$37\%$ of the 37921 instances are solved by IW(1) while $51.3\%$ are  solved by IW(2). Since IW($k$)
runs in time that is exponential in $k$, this mean that almost $90\%$ of the 37921 instances are  solved in time that
is either linear or quadratic in the number of problem variables, which in these encodings are all \emph{boolean}.  
Furthermore, when the performance of \IW\   is compared with  breadth-first search and iterative deepening, 
on the one hand,  and  with a  Greedy Best First Search guided by the additive heuritstic $h_{add}$
on the other (this algorithm is similar to one used by  the FF planner when hill climbing search fails \cite{hoffmann:ff}), 
it turns out that ``blind'' \IW\ solves as many problems as the informed search, 34627 vs. 34849, far ahead
of the other blind algorithms BrFS and ID  that solve 9010 and 8762 problems each. This is shown in the bottom
part of the Table~\ref{table1}. Moreover, \IW\ is  faster and results in shorter plans than the heuristic search  
\cite{nir:ecai2012}.


The min $k$ value for which $IW(k)$ solves an instance  is clearly bounded and small in most of
these instances. This is  actually no  accident and  has a \emph{theoretical explanation.} Lipovetzky and Geffner define
a structural parameter called the problem \emph{width} and show that for many of these domains,
\emph{any} solvable instance with atomic goals will  have a bounded and small width that is independent of the number
of variables and states in the problem. The min value $k$ for which the iteration  $IW(k)$ solves
the problem cannot exceed the problem width, so that the algorithm \emph{IW will run in time and space
that are exponential in the problem width.}

Formally, the \emph{width} $w(P)$ of a problem $P$ is $i$
iff $i$ is the minimum positive integer for which 
there is a sequence of tuples (sets) with at most $i$ atoms each, $t_0, t_1, \ldots, t_n$
such that 1)~$t_0$ is true in the initial state of $P$, 2)~any shortest plan $\pi$  that achieves $t_i$ in $P$
can be extended into a shortest plan that achieves $t_{i+1}$ by extending   $\pi$ with one action, and 
3)~any shortest plan that achieves $t_n$ is also a shortest plan for $P$. 

One way to understand this definition is that the problem  width $w(P)$ is at most  $i$ if  there is a 
``trail of  stepping stones''  $t_0, t_1, \ldots, t_n$ to reach the problem goal such that 
A)~these ``stepping stones''  contain at most $i$ atoms each, B)~each  stepping stone $t_{i+1}$ is 
at distance 1 from the previous one $t_i$ when $t_i$ has been reached following the ``trail'', 
C)~the ``trail''  preserves ``optimality''; i.e., no  tuple $t_i$ can be reached in less than $i$ steps. 

While this notion of width and the iterated  width algorithms that are based on it  have been designed
for problems where a goal state needs to be reached, the notions remain  relevant  in  optimization problems as well.
Indeed, if a  good path is made of states $s_i$ each of which have a low width, we would like 
IW to find such path in low polynomial time for a low value of the $k$ parameter. 
Later on we will discuss the changes required in IW to enforce this property, which
are not yet implemented in the code that is evaluated below.


\section{Width-Based Algorithms for the Atari Games}

Explain: IW(1), 2BFS, and possible IW11. Also, Dijkstra postprocessing over IW(1) graph
if ran.


The number of nodes \emph{generated} by IW(1) is $n \times D \times b$ in the worst case, 
where $n$ is the number of variables, $D$ is the size of their domains, and $b$ is maximum the number of 
actions applicable per  state.  This same number in breadth-first search is not linear in $n$
but exponential. For the Atari games,  $n=128$, $D=256$, and $b=18$, so that the product is
equal to $589,824$, which is large but feasible. On the other hand, the number
of nodes generated by IW(2) in the worst case is  $(n \times D)^2 \times b$, which is
equal to $1,073,741,824$ which is too large, forcing us to consider only a fraction of 
such states. For classical planning problems, the growth in the number of nodes from 
IW(1) to IW(2) is not that large, as the variables are boolean. Indeed, we could have
taken the state vector for the Atari games as a vector of 1024 boolean variables, 
and apply these algorithms to that representation. The results would be different.
In such a case, IW(1) would generate $n' \times D' \times b$ states, and 
IW(2), $(n' \times D') \times b$ states, which for $n'=1024$ and $D=2$
represent  $36,864$ and $7,5497,472$ states respectively.
These are both feasible numbers, but we haven't tried yet this representation,
under the assumption that the correlations among the bits in each one of 
the 128 words of the state vectors are meaningful. 
In summary, from the basic IW algorithm we are testing only the first call, IW(1).


IW is a purely exploration algorithm that does not take into account  the accumulated 
reward for selecting the next states to consider. As a simple, variant that combines
exploration and exploitation, we evaluated a \emph{best-first search} algorithm
with two queues: one queue  ordered first by  novelty measure 
(recall, novelty 1 means that the state is the first one to make some atom true),
and a second queue ordered by accumulated reward. In one iteration, the best first
search picks up the best node from one queue, and in the second iteration it picks
up the best node from the other queue. This way for combining multiple heuristics
is used in the LAMA planner \cite{richter:lama}, and was introduced in the
planner Fast Downward \cite{malte:fd,malte:alternating}. In addition, we break ties
in the first queue favoring states with largest accumulated reward, and 
in the second queue, favoring states with smallest novelty measure. Last, when
a node is expanded, it is removed from the queue, and its children are placed on 
both queues. The exception are the nodes with novelty 1 measure that are only
placed in first queue (** check and revise; but keep accurate and simple **).
We refer to this best-first algorithm below, as 2BFS.

IW11? 


Randomization? Discount?

\section{Experimental Results in the Atari Games}

\begin{table*}
%% \addtolength{\tabcolsep}{-2pt}
{\scriptsize
\begin{tabular}{@{}l|rr|rrr|rrr|rrr|rrr@{}}
%% table header
\hline\\[-3ex]
\multicolumn{1}{c|}{} & \multicolumn{2}{c|}{Random} & \multicolumn{3}{c|}{BrFS}& \multicolumn{3}{c|}{IW1}& \multicolumn{3}{c|}{BFS}& \multicolumn{3}{c}{UCT} \\[.2ex]
\hline\\[-3ex]
%% table body
\multicolumn{1}{c}{Game}  & \multicolumn{1}{c}{S} & \multicolumn{1}{c|}{Dev}  & \multicolumn{1}{c}{T} & \multicolumn{1}{c}{S} & \multicolumn{1}{c}{Dev}  & \multicolumn{1}{|c}{T} & \multicolumn{1}{c}{S} & \multicolumn{1}{c}{Dev}  & \multicolumn{1}{|c}{T} & \multicolumn{1}{c}{S} & \multicolumn{1}{c}{Dev}  & \multicolumn{1}{|c}{T} & \multicolumn{1}{c}{S} & \multicolumn{1}{c}{Dev} \\[.2ex]
\hline\\[-3ex]
asterix         & 210   & 54.77     & 32.24 & 3900  & 5538.95   & 17.08 & 119000        & 4987.48   & 0.95  & 56320       & 4757.84   & 10.82 & \textbf{296300} & 6657.7    \\
beam\_rider     & 448.8 & 84.64     & 43.48 & 519.2 & 179.81    & 6.02  & 3000          & 1150.22   & 0.92  & 3804        & 446.18    & 42.91 & \textbf{4320}   & 2175.26   \\
breakout        & 1     & 0.71      & 51.56 & 2     & 1.41      & 1.37  & 383.6         & 9.84      & 3.95  & 421.2       & 13.66     & 26.03 & \textbf{512.4}  & 193.97    \\
enduro          & 0     & 0         & 40.92 & 0.4   & 0.89      & 38.13 & \textbf{500}  & 0         & 13.12 & 292.8       & 6.18      & 32.14 & 306.8           & 8.14      \\
freeway         & 0     & 0         & 71.84 & 0     & 0         & 15.19 & \textbf{29.4} & 0.89      & 4.44  & 19.6        & 3.29      & 90.64 & 1               & 1.41      \\
pong            & -20.4 & 0.55      & 41.51 & -20.2 & 0.84      & 18.45 & 16.4          & 1.67      & 39.69 & \textbf{21} & 0         & 38.37 & \textbf{21}     & 0         \\
qbert           & 185   & 345.33    & 42.83 & 160   & 85.88     & 1.82  & 1545          & 1666.06   & 1.76  & 14725       & 7872.1    & 28.61 & \textbf{18730}  & 1127.69   \\
seaquest        & 192   & 38.99     & 40.01 & 200   & 14.14     & 14.72 & \textbf{4274} & 4206.81   & 1.59  & 4070        & 1939.55   & 26.16 & 3406            & 2402.47   \\
space\_invaders & 187   & 120.86    & 46.11 & 115   & 61.44     & 12.1  & \textbf{2670} & 637.3     & 2.25  & 1327        & 598.04    & 19.66 & 2634            & 375.66   
\end{tabular}
}

 \caption{\small
   Random, BrFS, IW1, BFS, UCT performance over 5 runs, with
   seeds 0 to 4.  \textit{T} is the action selection average
   computation time, \textit{S} is the average score, and
   \textit{Dev} is the standard deviation.  Numbers in bold show best performer in terms of average score.}
\label{performance}

\end{table*}


** Results, cold and detached, same format as those reported in jair paper, possibly with
std deviations and times. 

** mention also ideas that have been tried and didn't work (IW2?, discount, randomization, ..)

 
\section{Related and Futue Work}

- Relation  to Novelty-based search (Stanley)
- IW probes vs UCT probes; UCT using novelty with noise in rollout???
- Structured Exploration of  Factored Spaces
- Dijkstra postprocessing, properties for optimization
- Learning in same instance, across diff instances of same game, etc.
- Learning controllers vs. on-line planning, etc


\end{document}

